% Created 2018-08-06 Mo 21:27
\documentclass[a4paper,10pt,twoside]{article}
\usepackage[utf8]{inputenc}
\usepackage[T1]{fontenc}
\usepackage{graphicx}
\usepackage{longtable}
\usepackage{hyperref}
\usepackage{pdflscape}
\usepackage{fullpage}
\usepackage[utf8]{inputenc}
\usepackage[T1]{fontenc}
\usepackage{graphicx}
\usepackage{grffile}
\usepackage{longtable}
\usepackage{wrapfig}
\usepackage{rotating}
\usepackage[normalem]{ulem}
\usepackage{amsmath}
\usepackage{textcomp}
\usepackage{amssymb}
\usepackage{capt-of}
\usepackage{hyperref}
\usepackage{fancyhdr}
\usepackage{pdfpages}
\usepackage{tabulary}
\pagestyle{fancy}
\fancyhead{} % clear all fields
\fancyhead[RO,LE]{\thepage}
\fancyhead[C]{\slshape\nouppercase{\leftmark}}
\fancyfoot[RO,LE]{\thepage}
\fancyfoot[C]{\slshape Grundofen, Coswiger Str 1, 01445 Radebeul}
\setlength{\headsep}{10pt}
\addtolength{\headheight}{\baselineskip}
\renewcommand{\headrulewidth}{0.4pt}
\renewcommand{\footrulewidth}{0.4pt}
\author{Robert Hennig}
\date{2018-08-4}
\title{Berechnungen Grundofen Coswiger Str 1}
\hypersetup{
 pdfauthor={Robert Hennig},
 pdftitle={Berechnungen Grundofen Coswiger Str 1},
 pdfkeywords={},
 pdfsubject={},
 pdfcreator={Emacs 26.1 (Org mode 9.1.13)}, 
 pdflang={English}}
\begin{document}

\maketitle
\setcounter{tocdepth}{2}
\tableofcontents


\section{Grundofenberechnung nach EN 15544 \cite{EN15544}}
\label{sec:org6be199b}
\subsection{Bestimmung der Nennwärmeleistung (4.1)}
\label{sec:org5b09e06}
Die aktive Oberfläche des Ofens sind rund \(5 m²\).
Es wird ein mittelschwerer Grundofen errichtet,
dafür gilt nach \cite{Herrmann2011}, Seite 229 
\begin{align}
P_n & = A_{GO} \cdot q_{GO} \\
    & = 5 m^2 \cdot 0.75 kW/m^2 = 3.75 kW
\end{align}
\subsection{Bestimmung der Brennstoffmasse (4.2)}
\label{sec:org3cead4b}
\subsubsection{Maximale Brennstoffmasse (4.2.1)}
\label{sec:orgacf1b40}
Es eine Speicherdauer von \(t_n=8h\) vereinbart, so daß
\begin{align}
m_B & = \frac{P_n \cdot t_n}{3.25} \\
    & = \frac{3.75 \cdot 8}{3.25} & = 9.2 kg
\end{align}
\subsubsection{Minimale Brennstoffmasse (4.2.2)}
\label{sec:orgfc5f8d8}
\begin{equation}
m_{Bmin} = 0.5 \cdot m_B = 0.5 \cdot 9.2 kg = 4.6 kg
\end{equation}

\subsection{Festlegung der wesentlichen Abmessungen (4.3)}
\label{sec:org86a3b08}
\subsubsection{Brennraum-Innenfläche (4.3.1.1)}
\label{sec:orged00d8b}
\begin{equation}
O_{BR}= 900 \cdot m_{B} = 900 \cdot 9.2 = 8280 cm^2
\end{equation}
\subsubsection{Brennraum-Grundfläche (4.3.1.2)}
\label{sec:org4fda0f9}
\begin{equation}
A_{BRmin}=100 \cdot m_B = 100 \cdot 9.2 = 920 cm^2
\end{equation}
Es wird eine Brennraum Grundfläche von \(A_{BR} = 36 cm \times 36 cm = 1296 cm^2\) vereinbart.
\begin{align}
A_{BRmax}& =\frac{900 \cdot m_B - ( 25 + m_B) \cdot U_{BR}}{2} \\
         & =\frac{900 \cdot 9.2 - ( 25 + 9.2) \cdot 4 \cdot 36}{2} = 1677 cm^2
\end{align}
Es ist
\begin{align}
	A_{BRmin} &\le A_{BR} &\le A_{BRmax} \\
	920 cm^2  &\le 1296 cm^2  &\le 1677 cm^2
\end{align}
und die minimale Seitenbreite von \(23 cm\) wird eingehalten.
\subsubsection{Brennraumhöhe (4.3.1.3)}
\label{sec:org907690a}
\begin{align}
H_{BR} &\ge 25 + m_B = 25 + 9.2 = 34.2 cm \\
H_{BR} &= \frac{900\cdot m_B - 2 \cdot A_{BR}}{U_{BR}} = \frac{900\cdot 9.2 - 2 \cdot 36 \cdot 36}{4 \cdot 36} = 39.5 cm
\end{align}
Die Brennraumhöhe wird auf \(H_{BR} = 40 cm\) festgelegt.

\subsubsection{Mindestzuglänge (4.3.2.1)}
\label{sec:org8835802}
Der Grundofen wird ohne Luftspalt gebaut, dh die Mindestzuglänge ergibt sich zu
\begin{equation}
 L_{Zmin} = 1.3 \cdot \sqrt{m_B} = 1.3 \cdot \sqrt{9.2} = 3.94 m
\end{equation}
\subsubsection{Gasschlitzquerschnitt (4.3.3)}
\label{sec:orgf097d90}
\begin{equation}
A_{GS} = 1 \cdot m_B = 1 \cdot 9.2 = 9.2 cm^2
\end{equation}

\subsection{Verbrennungsluft, Verbrennungsgas}
\label{sec:org433f19b}
\subsubsection{Seehöhen-Korrektur (4.6.1.2)}
\label{sec:org8a1b0a4}
Die geodätische Höhe des Aufstellungsortes ist \$ z = 110 m \$.
\begin{equation}
f_s = \frac{1}{e^\frac{-9.81 \cdot z}{78624}} = \frac{1}{e^\frac{-9.81 \cdot 110}{78624}} = 1.0138 
\end{equation}

\subsection{Temperatur und Druckbedingungen in den Zügen}
\label{sec:org93bd527}
\subsubsection{Zuglaengen}
\label{sec:org94432d9}
Auf der nächsten Seite sind die Züge dargestellt und
in Tabelle \ref{tab:orge871124} aufgeführt.    
\includepdf[pagecommand={},lastpage=1]{zuglaengen.pdf}

\begin{table}[htbp]
\caption{\label{tab:orge871124}
Zuglängen und \(\zeta\) Werte}
\centering
\begin{tabular}{rrrrrrrr}
n & \(a\) & \(b\) & \(r\) & \(\alpha\) & \(L_Z\) & \(D_h\) & \(\zeta\)\\
 & cm & cm & cm & deg & cm & cm & \\
\hline
1 & 36 & 36 &  & 90 & 40 & 36 & 1.2\\
2 & 17 & 17 &  & 90 & 20 & 17 & 1.2\\
3 & 17 & 17 &  & 90 & 63 & 17 & 1.2\\
4 & 17 & 17 &  & 0 & 23 & 17 & 0\\
5 & 17 & 17 & 27 & 96 & 45 & 17 & 1\\
6 & 17 & 17 & 36 & 111 & 70 & 17 & 1.1\\
7 & 17 & 17 &  &  & 76 & 17 & 0\\
8 & 17 & 17 & 31 & 68 & 37 & 17 & 0.8\\
9 & 17 & 17 &  &  & 38 & 17 & 0\\
10 & 17 & 17 & 25 & 75 & 33 & 17 & 0.8\\
11 & 17 & 17 &  &  & 79 & 17 & 0\\
12 & 15 & 17 & 35 & 109 & 67 & 16 & 1.2\\
13 & 15 & 17 &  &  & 19 & 16 & 0\\
14 & 15 & 17 & 18 & 90 & 28 & 16 & 1.1\\
15 & 15 & 17 &  &  & 35 & 16 & 1.2\\
16 & 15 & 17 &  &  & 120 & 16 & 0.2\\
vst 17 & 17 & 17 & 12 & 90 & 19 & 17 & 0.4\\
vst 18 & 17 & 17 &  &  & 26 & 17 & 1.2\\
\end{tabular}
\end{table}

\subsubsection{Temperatur und Druckverlauf in den Zügen}
\label{sec:org6897b8e}
In Tabelle \ref{tab:org90e86ff} ist der Temperatur und Druckverlauf in den Zügen dargestellt.
Der Temperaturverlauf in Verbindungsstück und Schornstein kommt aus den Berechnungen 
aus Abschnitt \ref{org0cbeb1d}

\begin{landscape}

\begin{table}[htbp]
\caption{\label{tab:org90e86ff}
Temperatur und Druckverlauf in den Zügen}
\centering
\small
\begin{tabulary}{\textwidth}{rrrrrrrrrrrrrrrrrrrr}
Zug & \(L_Z\) & \(H\) & \(a\) & \(b\) & \(\zeta\) & \(\sum L_Z\) & \(t\) & \(f_t\) & \(\dot{V_G}\) & \(\rho_G\) & \(p_h\) & \(A\) & \(U\) & \(D_h\) & \(v\) & \(p_d\) & \(\lambda_f\) & \(p_r\) & \(p_u\)\\
 & m & m & cm & cm &  & m & °C &  & m³/s & kg/m³ & Pa & m² & m & m & 1.2..6m/s & Pa &  & Pa & Pa\\
 &  &  &  &  &  &  & 4.8.2 & 4.6.1.1 & 4.6.2 & 4.7.2 & 4.9.1 &  &  & 4.9.3.4 & 4.9.2 & 4.9.3.2 & 4.9.3.3 & 4.9.3.1 & 4.9.4\\
\hline
 &  &  &  &  &  &  &  &  &  &  &  &  &  &  &  &  &  &  & 4\\
 & 0.05 & 0 & 10 & 4 & 0.04 & 0.05 & 18 & 1.066 & 0.027 & 1.186 & 0.00 & 0.004 & 0.28 & 0.057 & 6.750 & 27.019 & 0.073 & 1.730 & 1.081\\
\hline
1 & 0.40 & .40 & 36 & 36 & 1.2 & 0.45 & 700 & 3.564 & 0.091 & 0.355 & 3.57 & 0.130 & 1.44 & 0.361 & 0.700 & 0.087 & 0.036 & 0.003 & 0.104\\
\hline
2 & .20 & 0 & 16 & 10 & 1.2 & 0.65 & 495 & 2.813 & 0.072 & 0.450 & 0.00 & 0.016 & 0.52 & 0.123 & 4.500 & 4.556 & 0.052 & 0.385 & 5.467\\
3 & .63 & -.63 & 17 & 17 & 1.2 & 1.28 & 454 & 2.663 & 0.068 & 0.475 & -4.88 & 0.029 & 0.68 & 0.171 & 2.345 & 1.306 & 0.046 & 0.221 & 1.567\\
4 & .23 & 0 & 17 & 17 & 0 & 1.51 & 414 & 2.516 & 0.064 & 0.503 & 0.00 & 0.029 & 0.68 & 0.171 & 2.207 & 1.225 & 0.046 & 0.076 & 0.000\\
5 & .45 & 0 & 17 & 17 & 1.0 & 1.96 & 386 & 2.414 & 0.061 & 0.524 & 0.00 & 0.029 & 0.68 & 0.171 & 2.103 & 1.159 & 0.046 & 0.140 & 1.159\\
6 & .70 & 0 & 17 & 17 & 1.1 & 2.66 & 342 & 2.253 & 0.057 & 0.561 & 0.00 & 0.029 & 0.68 & 0.171 & 1.966 & 1.084 & 0.046 & 0.204 & 1.192\\
7 & .76 & 0 & 17 & 17 & 0 & 3.42 & 293 & 2.073 & 0.053 & 0.610 & 0.00 & 0.029 & 0.68 & 0.171 & 1.828 & 1.019 & 0.046 & 0.208 & 0.000\\
8 & .37 & 0 & 17 & 17 & 0.8 & 3.79 & 260 & 1.952 & 0.050 & 0.648 & 0.00 & 0.029 & 0.68 & 0.171 & 1.724 & 0.963 & 0.046 & 0.096 & 0.770\\
9 & .38 & 0 & 17 & 17 & 0 & 4.17 & 240 & 1.879 & 0.048 & 0.673 & 0.00 & 0.029 & 0.68 & 0.171 & 1.655 & 0.922 & 0.046 & 0.094 & 0.000\\
10 & .33 & 0 & 17 & 17 & 0.8 & 4.5 & 223 & 1.817 & 0.046 & 0.696 & 0.00 & 0.029 & 0.68 & 0.171 & 1.586 & 0.875 & 0.046 & 0.078 & 0.700\\
11 & .79 & 0 & 16 & 17 & 0 & 5.29 & 198 & 1.725 & 0.044 & 0.733 & 0.00 & 0.027 & 0.66 & 0.164 & 1.630 & 0.974 & 0.047 & 0.221 & 0.000\\
12 & .67 & 0 & 15 & 17 & 1.2 & 5.96 & 170 & 1.623 & 0.041 & 0.779 & 0.00 & 0.025 & 0.64 & 0.156 & 1.640 & 1.048 & 0.048 & 0.216 & 1.258\\
13 & .19 & 0 & 15 & 17 & 0 & 6.15 & 155 & 1.568 & 0.040 & 0.806 & 0.00 & 0.025 & 0.64 & 0.156 & 1.600 & 1.032 & 0.048 & 0.060 & 0.000\\
14 & .28 & 0 & 15 & 17 & 1.1 & 6.43 & 148 & 1.542 & 0.039 & 0.820 & 0.00 & 0.025 & 0.64 & 0.156 & 1.560 & 0.998 & 0.048 & 0.086 & 1.098\\
15 & .35 & 0 & 15 & 17 & 1.2 & 6.78 & 138 & 1.505 & 0.038 & 0.840 & 0.00 & 0.025 & 0.64 & 0.156 & 1.520 & 0.970 & 0.048 & 0.104 & 1.164\\
16 & 1.2 & 1.2 & 15 & 17 & 0.2 & 7.98 & 117 & 1.429 & 0.036 & 0.885 & 4.46 & 0.025 & 0.64 & 0.156 & 1.440 & 0.918 & 0.048 & 0.339 & 0.184\\
\hline
vst & 0.45 & 0.12 & 17.0 &  & 1.4 & 8.43 & 115 & 1.421 & 0.036 & 0.890 & 0.44 & 0.023 & 0.53 & 0.174 & 1.565 & 1.090 & 0.040 & 0.113 & 1.526\\
sch & 7.1 & 7.1 & 17.8 &  & 0.01 & 15.53 & 106 & 1.388 & 0.035 & 0.911 & 24.59 & 0.025 & 0.56 & 0.179 & 1.400 & 0.893 & 0.039 & 1.381 & 0.009\\
\hline
 &  &  &  &  &  & 15.53 & 98 &  &  &  & 28.18 &  & 0.00 &  &  & 48.138 & 0.000 & 5.755 & 21.279\\
\end{tabulary}
\end{table}

\end{landscape}

\subsubsection{Funktionskontrolle (4.10)}
\label{sec:org8413c79}

\begin{table}[htbp]
\caption{\label{tab:orgaee0935}
Druckvergleich, Wirkungsgrad}
\centering
\begin{tabular}{llrlr}
\(\sum p_r+ \sum p_u \le\) & \(\sum p_h\) & \(\le 1.05\cdot (\sum p_r+ \sum p_u)\) & \(t_F\) & \(\eta\)\\
Pa & Pa & Pa & °C & \\
 &  & 4.10.1 &  & 4.10.3\\
\hline
27.03 & 28.18 & 28.38 & 117 & 89.944\\
 &  &  &  & \\
\end{tabular}
\end{table}

In der Tabelle \ref{tab:orgaee0935} sind die Ergebnisse dargestellt.

\section{Berechnung Abgasanlage nach EN 13384-1 \cite{EN13384}}
\label{sec:org2d528ce}
\label{org0cbeb1d}
\subsection{Wärmedurchlasswiderstand für den Schornstein (5.6.3)}
\label{sec:org09a5b82}
\begin{table}[htbp]
\caption{Wärmedurchlasswiderstand für Schornstein}
\centering
\begin{tabular}{lrrr}
Schicht & \(D_h\) & \(\lambda\) & \(1/\Lambda\)\\
 & \(m\) & \(W/(m\cdot K)\) & \(m^2\cdot K/W\)\\
 &  & B.5 & A.1\\
\hline
Keramikrohr & 0.178 & 1.02 & 0.010\\
Mineralwolle & 0.20 & 0.042 & 0.386\\
 & 0.24 &  & \\
 &  &  & 0.396\\
\end{tabular}
\end{table}

\subsection{Taupunkttemperatur 5.7.6}
\label{sec:orga2b9931}
\begin{table}[htbp]
\caption{Taupunkttemperatur}
\centering
\begin{tabular}{llllllrll}
Brennstoff & \(f_{m1}\) & \(f_W\) & \(\sigma(H_2O)\) & \(z\) & \(T\) & \(p_L\) & \(p_D\) & \(t_p\)\\
 &  & \% & \% & m & °C & Pa & Pa & °C\\
\(\sigma(CO_2)=20.5\) & B.1 & B.1 & B.5 &  &  & 5.7.2 & B.6 & B.7\\
\hline
Holz 30\% feucht & 6.89 & 90 & 19.652 & 119 & 0 & 95570 & 18781 & 58.7\\
Holz 50\% feucht & 7.08 & 72 & 23.262 & 119 & 0 & 95570 & 22231 & 62.4\\
\end{tabular}
\end{table}

\begin{landscape}
\subsection{Temperaturwerte 5.8}
\label{sec:org88da0bb}


\begin{table}[htbp]
\caption{\label{tab:org1722d04}
Berechnung der Abkühlzahl K}
\centering
\footnotesize
\begin{tabulary}{\textwidth}{llllllllllllllllllll}
 &  & \(\lambda_A\) & \(L\) & \(D_h\) & \(\eta_A\) & \(c_p\) & \(Pr\) & \(R\) & \(\rho_m\) & \(w_m\) & \(Re\) & \(\Psi\) & \(\Psi_s\) & \(Nu\) & \(\alpha_i\) & \(\frac{1}{\Lambda}\) & \(D_{ha}\) & \(k\) & \(K\)\\
 & °C & \(\frac{W}{m\cdot K}\) & \(m\) & \(m\) & \(N\cdot s/m^2\) & \(\frac{J}{kg\cdot K}\) &  & \(\frac{J}{kg\cdot K}\) & \(kg/m³\) & \(m/s\) &  &  &  &  & \(\frac{W}{m²\cdot K}\) & \(\frac{m^2 \cdot K}{W}\) & \(m\) & \(\frac{W}{m^2\cdot K}\) & \\
 &  & (B.9) &  &  & (B.10) & (B.4) & (25) & (B.3) &  &  & (26) & (35) & (35) & (24) & (23) &  &  & (21) & (20)\\
\hline
vst & 115 & 0.0298 & 0.45 & 0.170 & 2.014\,(-05) & 1209 & 0.817 & 288.590 & 0.854 & 1.662 & 11981 & 0.0411 & 0.0295 & 64.91 & 11.378 & 0.01 & 0.2 & 4.899 & 0.030\\
sch & 106 & 0.0292 & 7.1 & 0.178 & 1.976\,(-05) & 1206 & 0.816 & 288.590 & 0.874 & 1.481 & 11660 & 0.0408 & 0.0297 & 44.79 & 7.348 & 0.396 & 0.24 & 1.601 & 0.164\\
\end{tabulary}
\end{table}

\begin{table}[htbp]
\caption{\label{tab:org04d27a5}
Rohrreibungszahl \(\Psi\) nach 5.10.3.3 (35)}
\centering
\begin{tabular}{lrrrrr}
 & \(Re\) & \(D_h\) & \(r\) & \(\frac{1}{\sqrt{\Psi}}\) & \(\Psi\)\\
 &  & m & m &  & \\
\hline
\(vst\) & 11981 & 0.17 & 0.0015 & 4.9339890 & 0.041077466\\
\(vst_{smooth}\) & 11981 & 0.17 & 0 & 5.8267813 & 0.029453883\\
\(sch\) & 11660 & 0.178 & 0.0015 & 4.9531167 & 0.040760817\\
\(sch_{smooth}\) & 11660 & 0.178 & 0 & 5.8062571 & 0.029662481\\
\end{tabular}
\end{table}

\begin{table}[htbp]
\caption{\label{tab:org023034d}
Temperaturen in Verbindungsstück und Schornstein}
\centering
\begin{tabular}{lllllllllllll}
\(t_w\) & \(t_u\) & \(t_{uv}\) & \(T_w\) & \(T_u\) & \(T_{uv}\) & \(T_e\) & \(T_{mv}\) & \(T_o\) & \(T_m\) & \(t_{mv}\) & \(t_m\) & \(t_o\)\\
°C & °C & °C & K & K & K & K & K & K &  & °C & °C & °C\\
 &  &  &  & (11) & (11) & (19) & (18) & (17) & (16) &  &  & \\
\hline
117 & 8 & 20 & 390 & 281 & 293 & 386.8 & 388.4 & 370.8 & 378.6 & 115 & 106 & 98\\
\end{tabular}
\end{table}

\end{landscape}

\subsection{Ermittlung der Drücke 5.10}
\label{sec:org6a06f11}
\begin{table}[htbp]
\caption{\label{tab:orgbcdd843}
Drücke im Schornstein}
\centering
\begin{tabular}{llllllllllllll}
\(pL\) & \(\rho_L\) & \(\rho_m\) & \(H\) & \(D_h\) & \(P_H\) & \(\Psi\) & \(w_m\) & \(P_R\) & \(Q_N\) & \(P_W\) & \(P_B\) & \(P_Z\ge\) & \(P_{Ze}\)\\
\(Pa\) & \(kg/m^3\) & \(kg/m^3\) & \(m\) & \(m\) & Pa &  & \(m/s\) & Pa & \(kW\) & Pa & Pa & Pa & Pa\\
 &  & (27) &  &  & (31) &  &  & (33) &  & B.2 &  & (1) & (1)\\
\hline
95570 & 1.216 & 0.875 & 7.1 & 0.178 & 23.8 & 0.0408 & 0.816 & 0.710 & 3.75 & 8.6 & 4 & 23.09 & 12.6\\
\end{tabular}
\end{table}


Die einzuhaltende Bedingung ist nach (1),5.2.1:
\begin{align}
  P_Z = P_H - P_R \ge P_W + P_{FV} + P_B = P_{Ze}
\end{align}








\bibliography{citations}
\bibliographystyle{alpha}
\end{document}
